% Options for packages loaded elsewhere
\PassOptionsToPackage{unicode}{hyperref}
\PassOptionsToPackage{hyphens}{url}
\PassOptionsToPackage{dvipsnames,svgnames,x11names}{xcolor}
%
\documentclass[
  letterpaper,
  DIV=11,
  numbers=noendperiod]{scrreprt}

\usepackage{amsmath,amssymb}
\usepackage{iftex}
\ifPDFTeX
  \usepackage[T1]{fontenc}
  \usepackage[utf8]{inputenc}
  \usepackage{textcomp} % provide euro and other symbols
\else % if luatex or xetex
  \usepackage{unicode-math}
  \defaultfontfeatures{Scale=MatchLowercase}
  \defaultfontfeatures[\rmfamily]{Ligatures=TeX,Scale=1}
\fi
\usepackage{lmodern}
\ifPDFTeX\else  
    % xetex/luatex font selection
\fi
% Use upquote if available, for straight quotes in verbatim environments
\IfFileExists{upquote.sty}{\usepackage{upquote}}{}
\IfFileExists{microtype.sty}{% use microtype if available
  \usepackage[]{microtype}
  \UseMicrotypeSet[protrusion]{basicmath} % disable protrusion for tt fonts
}{}
\makeatletter
\@ifundefined{KOMAClassName}{% if non-KOMA class
  \IfFileExists{parskip.sty}{%
    \usepackage{parskip}
  }{% else
    \setlength{\parindent}{0pt}
    \setlength{\parskip}{6pt plus 2pt minus 1pt}}
}{% if KOMA class
  \KOMAoptions{parskip=half}}
\makeatother
\usepackage{xcolor}
\setlength{\emergencystretch}{3em} % prevent overfull lines
\setcounter{secnumdepth}{5}
% Make \paragraph and \subparagraph free-standing
\ifx\paragraph\undefined\else
  \let\oldparagraph\paragraph
  \renewcommand{\paragraph}[1]{\oldparagraph{#1}\mbox{}}
\fi
\ifx\subparagraph\undefined\else
  \let\oldsubparagraph\subparagraph
  \renewcommand{\subparagraph}[1]{\oldsubparagraph{#1}\mbox{}}
\fi


\providecommand{\tightlist}{%
  \setlength{\itemsep}{0pt}\setlength{\parskip}{0pt}}\usepackage{longtable,booktabs,array}
\usepackage{calc} % for calculating minipage widths
% Correct order of tables after \paragraph or \subparagraph
\usepackage{etoolbox}
\makeatletter
\patchcmd\longtable{\par}{\if@noskipsec\mbox{}\fi\par}{}{}
\makeatother
% Allow footnotes in longtable head/foot
\IfFileExists{footnotehyper.sty}{\usepackage{footnotehyper}}{\usepackage{footnote}}
\makesavenoteenv{longtable}
\usepackage{graphicx}
\makeatletter
\def\maxwidth{\ifdim\Gin@nat@width>\linewidth\linewidth\else\Gin@nat@width\fi}
\def\maxheight{\ifdim\Gin@nat@height>\textheight\textheight\else\Gin@nat@height\fi}
\makeatother
% Scale images if necessary, so that they will not overflow the page
% margins by default, and it is still possible to overwrite the defaults
% using explicit options in \includegraphics[width, height, ...]{}
\setkeys{Gin}{width=\maxwidth,height=\maxheight,keepaspectratio}
% Set default figure placement to htbp
\makeatletter
\def\fps@figure{htbp}
\makeatother
% definitions for citeproc citations
\NewDocumentCommand\citeproctext{}{}
\NewDocumentCommand\citeproc{mm}{%
  \begingroup\def\citeproctext{#2}\cite{#1}\endgroup}
\makeatletter
 % allow citations to break across lines
 \let\@cite@ofmt\@firstofone
 % avoid brackets around text for \cite:
 \def\@biblabel#1{}
 \def\@cite#1#2{{#1\if@tempswa , #2\fi}}
\makeatother
\newlength{\cslhangindent}
\setlength{\cslhangindent}{1.5em}
\newlength{\csllabelwidth}
\setlength{\csllabelwidth}{3em}
\newenvironment{CSLReferences}[2] % #1 hanging-indent, #2 entry-spacing
 {\begin{list}{}{%
  \setlength{\itemindent}{0pt}
  \setlength{\leftmargin}{0pt}
  \setlength{\parsep}{0pt}
  % turn on hanging indent if param 1 is 1
  \ifodd #1
   \setlength{\leftmargin}{\cslhangindent}
   \setlength{\itemindent}{-1\cslhangindent}
  \fi
  % set entry spacing
  \setlength{\itemsep}{#2\baselineskip}}}
 {\end{list}}
\usepackage{calc}
\newcommand{\CSLBlock}[1]{\hfill\break\parbox[t]{\linewidth}{\strut\ignorespaces#1\strut}}
\newcommand{\CSLLeftMargin}[1]{\parbox[t]{\csllabelwidth}{\strut#1\strut}}
\newcommand{\CSLRightInline}[1]{\parbox[t]{\linewidth - \csllabelwidth}{\strut#1\strut}}
\newcommand{\CSLIndent}[1]{\hspace{\cslhangindent}#1}

\KOMAoption{captions}{tableheading}
\makeatletter
\@ifpackageloaded{bookmark}{}{\usepackage{bookmark}}
\makeatother
\makeatletter
\@ifpackageloaded{caption}{}{\usepackage{caption}}
\AtBeginDocument{%
\ifdefined\contentsname
  \renewcommand*\contentsname{Table of contents}
\else
  \newcommand\contentsname{Table of contents}
\fi
\ifdefined\listfigurename
  \renewcommand*\listfigurename{List of Figures}
\else
  \newcommand\listfigurename{List of Figures}
\fi
\ifdefined\listtablename
  \renewcommand*\listtablename{List of Tables}
\else
  \newcommand\listtablename{List of Tables}
\fi
\ifdefined\figurename
  \renewcommand*\figurename{Figure}
\else
  \newcommand\figurename{Figure}
\fi
\ifdefined\tablename
  \renewcommand*\tablename{Table}
\else
  \newcommand\tablename{Table}
\fi
}
\@ifpackageloaded{float}{}{\usepackage{float}}
\floatstyle{ruled}
\@ifundefined{c@chapter}{\newfloat{codelisting}{h}{lop}}{\newfloat{codelisting}{h}{lop}[chapter]}
\floatname{codelisting}{Listing}
\newcommand*\listoflistings{\listof{codelisting}{List of Listings}}
\makeatother
\makeatletter
\makeatother
\makeatletter
\@ifpackageloaded{caption}{}{\usepackage{caption}}
\@ifpackageloaded{subcaption}{}{\usepackage{subcaption}}
\makeatother
\ifLuaTeX
  \usepackage{selnolig}  % disable illegal ligatures
\fi
\usepackage{bookmark}

\IfFileExists{xurl.sty}{\usepackage{xurl}}{} % add URL line breaks if available
\urlstyle{same} % disable monospaced font for URLs
\hypersetup{
  pdftitle={pQIT Notes Autumn '23},
  pdfauthor={Prasanna Paithankar},
  colorlinks=true,
  linkcolor={blue},
  filecolor={Maroon},
  citecolor={Blue},
  urlcolor={Blue},
  pdfcreator={LaTeX via pandoc}}

\title{pQIT Notes Autumn '23}
\author{Prasanna Paithankar}
\date{2025-05-12}

\begin{document}
\maketitle

\renewcommand*\contentsname{Table of contents}
{
\hypersetup{linkcolor=}
\setcounter{tocdepth}{2}
\tableofcontents
}
\bookmarksetup{startatroot}

\chapter*{About}\label{about}
\addcontentsline{toc}{chapter}{About}

\markboth{About}{About}

These notes were compiled from the lectures of \textbf{photonic Quantum
Information Techonlogies (AT60205) Autumn 2023} delivered by
\href{https://www.iitkgp.ac.in/department/AT/faculty/at-syamsundarde}{Prof.~Syamsunder
De} and
\href{https://scholar.google.co.in/citations?user=im2WPmIAAAAJ&hl=en}{Prof.~Fabien
Bretenaker} at \href{https://www.iitkgp.ac.in/}{Indian Institute of
Technology, Kharagpur}. The syllabus of the course can be found
\href{./Syllabus.pdf}{here}.

The course content is based on the book Introductory Quantum Optics by
Gerry and Knight (2005).

These notes by no means are complete and are not a substitute for the
lectures. They are meant to be used as a reference for the course. If
you find any errors, please feel free to open an issue or a pull
request.

\subsection*{Resources}\label{resources}
\addcontentsline{toc}{subsection}{Resources}

\begin{enumerate}
\def\labelenumi{\arabic{enumi}.}
\tightlist
\item
  \href{}{Practice Problems}
\item
  \href{}{Midsem Paper}
\item
  \href{}{Endsem Paper}
\end{enumerate}

\bookmarksetup{startatroot}

\chapter*{Basics of Quantum Physics \& Quantum Information
Science}\label{basics-of-quantum-physics-quantum-information-science}
\addcontentsline{toc}{chapter}{Basics of Quantum Physics \& Quantum
Information Science}

\markboth{Basics of Quantum Physics \& Quantum Information
Science}{Basics of Quantum Physics \& Quantum Information Science}

\section*{Dirac Formalism}\label{dirac-formalism}
\addcontentsline{toc}{section}{Dirac Formalism}

\markright{Dirac Formalism}

\(|x\rangle\) is called a ket viz.~a vector with complex values in
Hilbert space. It's dual is called a bra, \(\langle x|\)

Let \(\{|\phi_1\rangle, |\phi_2\rangle, ..., \phi_n\rangle\}\) be a
basis for a Hilbert space, then any vector \(|\psi\rangle\) can be
written as a linear combination of the basis vectors:

\[ |\psi\rangle = \sum_{i=1}^n c_i |\phi_i\rangle\ ;\ c_i \in \mathbb{C} \]

We would in this course consider 2-dimensional Hilbert space with
orthonormal basis \(\{|\phi_1\rangle, |\phi_2\rangle\}\) unless
otherwise stated.

If the basis follows \(\langle \phi_i | \phi_j \rangle = \delta_{ij}\)
then they are orthonormal. Furthermore
\(\sum_{i} |\phi_i\rangle \langle \phi_i | = I\) states that the basis
is complete.

\subsection*{Block Sphere
Representation}\label{block-sphere-representation}
\addcontentsline{toc}{subsection}{Block Sphere Representation}

Any state \(|\psi\rangle\) can be represented as a point on the block
sphere. The state \(|\psi\rangle\) is represented by the vector
\(\vec{r} = (r, \theta, \phi)\) where a general state is given by:

\[ |\psi\rangle = e^{i\gamma} \cos(\frac{\theta}{2}) |0\rangle + e^{i\gamma}e^{i\phi} \sin(\frac{\theta}{2}) |1\rangle \]

\(\gamma\) is the global phase.

\subsection*{Operators}\label{operators}
\addcontentsline{toc}{subsection}{Operators}

An operator is a linear map from one Hilbert space to another. An
operator \(\hat{A}\) acting on a state \(|\psi\rangle\) gives another
state \(|\psi'\rangle\).

\[\hat{A}|\psi\rangle = |\psi'\rangle\]

The operator \(\hat{A}\) can be represented as a matrix in the basis
\(\{|\phi_1\rangle, |\phi_2\rangle, ..., \phi_n\rangle\}\) as:

\[\hat{A} = \sum_{i,j} A_{ij} |\phi_i\rangle \langle \phi_j |\]

The matrix elements \(A_{ij}\) are given by:

\[A_{ij} = \langle \phi_i | \hat{A} | \phi_j \rangle\]

The operator \(\hat{A}\) is said to be Hermitian if
\(\hat{A} = \hat{A}^\dagger\) where \(\hat{A}^\dagger\) is the adjoint
of \(\hat{A}\).

\[\hat{A}^\dagger = \sum_{i,j} A_{ij}^* |\phi_j\rangle \langle \phi_i |\]

The adjoint of an operator is the transpose of the matrix elements with
complex conjugate.

Eigenvalues of an operator \(\hat{A}\) are the values \(\lambda\) such
that:

\[\hat{A}|\psi\rangle = \lambda|\psi\rangle\]

The eigenvalues of a Hermitian operator are real.
i.e.~\(\lambda \in \mathbb{R}\)

Unitary operators are those that preserve the norm of the state.
i.e.~\(\hat{U}^\dagger \hat{U} = \hat{U} \hat{U}^\dagger = I\)

\[\hat{U}(t) = e^{-i\hat{H}t/\hbar}\]

The operator \(\hat{H}\) is called the Hamiltonian and is Hermitian. The
Hamiltonian is the energy operator.

Commutator of two operators \(\hat{A}\) and \(\hat{B}\) is defined as:

\[[\hat{A}, \hat{B}] = \hat{A}\hat{B} - \hat{B}\hat{A}\]

\([\hat{A}, \hat{B}] = 0\) if \(\hat{A}\) and \(\hat{B}\) commute
i.e.~\(\hat{A}\hat{B} = \hat{B}\hat{A}\)

Uncertainty principle: \([\hat{A}, \hat{B}] \neq 0\) implies that the
operators do not commute and the uncertainty in the measurement of
\(\hat{A}\) and \(\hat{B}\) cannot be simultaneously zero.

\[\Delta A \Delta B \geq \frac{1}{2} |\langle [\hat{A}, \hat{B}] \rangle|\]

where
\(\Delta A = \sqrt{\langle \hat{A}^2 \rangle - \langle \hat{A} \rangle^2} = \sqrt{\langle \psi | \hat{A}^2 | \psi \rangle - \langle \psi | \hat{A} | \psi \rangle^2}\)

\subsection*{Pauli Matrices}\label{pauli-matrices}
\addcontentsline{toc}{subsection}{Pauli Matrices}

Defining \(|0\rangle = \begin{bmatrix} 1 \\ 0 \end{bmatrix}\) and
\(|1\rangle = \begin{bmatrix} 0 \\ 1 \end{bmatrix}\)

The Pauli matrices are defined as:

Identity
\[ I = \begin{bmatrix} 1 & 0 \\ 0 & 1 \end{bmatrix} = |0\rangle \langle 0 | + |1\rangle \langle 1 | \]

Pauli-X
\[ \sigma_x = \begin{bmatrix} 0 & 1 \\ 1 & 0 \end{bmatrix} = |0\rangle \langle 1 | + |1\rangle \langle 0 | \]

Pauli-Y
\[ \sigma_y = \begin{bmatrix} 0 & -i \\ i & 0 \end{bmatrix} = -i|0\rangle \langle 1 | + i|1\rangle \langle 0 | \]

Pauli-Z
\[ \sigma_z = \begin{bmatrix} 1 & 0 \\ 0 & -1 \end{bmatrix} = |0\rangle \langle 0 | - |1\rangle \langle 1 | \]

The Pauli matrices are Hermitian and unitary. The Pauli matrices are
also traceless. They do not commute with each other.

\([\sigma_x, \sigma_y] = 2i\sigma_z\ ;\ [\sigma_y, \sigma_z] = 2i\sigma_x\ ;\ [\sigma_z, \sigma_x] = 2i\sigma_y\)

On Bloch sphere, \(\sigma_x\) is rotation about \(x\)-axis by \(\pi\)
radians, \(\sigma_y\) is rotation about \(y\)-axis by \(\pi\) radians
and \(\sigma_z\) is rotation about \(z\)-axis by \(\pi\) radians.

\[ \begin{bmatrix} \langle \sigma_x \rangle \\ \langle \sigma_y \rangle \\ \langle \sigma_z \rangle \end{bmatrix} = \begin{bmatrix} \langle \psi | \sigma_x | \psi \rangle \\ \langle \psi | \sigma_y | \psi \rangle \\ \langle \psi | \sigma_z | \psi \rangle \end{bmatrix} = \begin{bmatrix} \cos(\phi)\sin(\theta) \\ \sin(\phi)\sin(\theta) \\ \cos(\theta) \end{bmatrix} \]

where
\(|\psi\rangle = \cos(\frac{\theta}{2}) |0\rangle + e^{i\phi} \sin(\frac{\theta}{2}) |1\rangle\)

Eigenvectors of Pauli matrices are:

\[\sigma_x |+\rangle = |+\rangle\ ;\ \sigma_x |-\rangle = -|-\rangle\]
\[\sigma_y |R\rangle = |R\rangle\ ;\ \sigma_y |L\rangle = -|L\rangle\]
\[\sigma_z |0\rangle = |0\rangle\ ;\ \sigma_z |1\rangle = -|1\rangle\]

where
\(|+\rangle = \frac{1}{\sqrt{2}} (|0\rangle + |1\rangle)\ ;\ |-\rangle = \frac{1}{\sqrt{2}} (|0\rangle - |1\rangle)\ ;\ |R\rangle = \frac{1}{\sqrt{2}} (|0\rangle + i|1\rangle)\)
and \(|L\rangle = \frac{1}{\sqrt{2}} (|0\rangle - i|1\rangle)\)

The transformations can be represented as:
\[(r, \theta, \phi) \overset{\sigma_x}{\rightarrow} (r, \theta, \pi + \phi)\]
\[(r, \theta, \phi) \overset{\sigma_y}{\rightarrow} (r, \pi - \theta, -\phi)\]
\[(r, \theta, \phi) \overset{\sigma_z}{\rightarrow} (r, \pi - \theta, -\pi - \phi)\]

\subsection*{Single Photon
Polarization}\label{single-photon-polarization}
\addcontentsline{toc}{subsection}{Single Photon Polarization}

Let \$\textbar H\rangle \equiv \textbar0\rangle \$ and
\(|V\rangle \equiv |1\rangle\) be the horizontal and vertical
polarization states of a photon.

The diagonal polarization states are
\(|D\rangle = \frac{1}{\sqrt{2}} (|H\rangle + |V\rangle)\) and
\(|A\rangle = \frac{1}{\sqrt{2}} (|H\rangle - |V\rangle)\)

The circular polarization states are
\(|R\rangle = \frac{1}{\sqrt{2}} (|H\rangle + i|V\rangle)\) and
\(|L\rangle = \frac{1}{\sqrt{2}} (|H\rangle - i|V\rangle)\)

\subsection*{Effect of optical elements on polarization
states:}\label{effect-of-optical-elements-on-polarization-states}
\addcontentsline{toc}{subsection}{Effect of optical elements on
polarization states:}

\subsubsection*{Linear Polarizer}\label{linear-polarizer}
\addcontentsline{toc}{subsubsection}{Linear Polarizer}

A linear polarizer transmits light polarized along a particular
direction. The transmission axis of the polarizer is the direction of
polarization of the transmitted light.

\[|\psi\rangle \overset{P}{\rightarrow} |\psi'\rangle\]

where \(|\psi\rangle = \alpha|H\rangle + \beta|V\rangle\) and
\(|\psi'\rangle = \alpha|H\rangle\)

\[ P = \begin{bmatrix} \cos^2(\theta) & \cos(\theta)\sin(\theta) \\ \cos(\theta)\sin(\theta) & \sin^2(\theta) \end{bmatrix} \]

where \(\theta\) is the angle between the transmission axis of the
polarizer and the horizontal axis.

\subsubsection*{Quarter Wave Plate}\label{quarter-wave-plate}
\addcontentsline{toc}{subsubsection}{Quarter Wave Plate}

A quarter wave plate converts linear polarization to circular
polarization and vice versa. The quarter wave plate is oriented at an
angle of \(\theta\) with respect to the horizontal axis.

\[ QWP = \begin{bmatrix} \exp(-i\pi/4) & 0 \\ 0 & \exp(i\pi/4) \end{bmatrix} = \frac{1}{\sqrt{2}} \begin{bmatrix} 1 & -i \\ -i & 1 \end{bmatrix} \]

\subsubsection*{Half Wave Plate}\label{half-wave-plate}
\addcontentsline{toc}{subsubsection}{Half Wave Plate}

A half wave plate converts linear polarization to linear polarization
and vice versa. The half wave plate is oriented at an angle of
\(\theta\) with respect to the horizontal axis.

\[HWP = \sigma_z\]

\subsection*{More Operators}\label{more-operators}
\addcontentsline{toc}{subsection}{More Operators}

\subsubsection*{Trace operator}\label{trace-operator}
\addcontentsline{toc}{subsubsection}{Trace operator}

The trace operator is defined as:

\[ Tr(\hat{A}) = \sum_i \langle \phi_i | \hat{A} | \phi_i \rangle \]

The inner product of two states \(|\psi\rangle\) and \(|\phi\rangle\)
can be given by:

\[\langle \psi | \phi \rangle = Tr(|\psi\rangle \langle \phi|)\]

\subsubsection*{Density Operator}\label{density-operator}
\addcontentsline{toc}{subsubsection}{Density Operator}

The density operator is defined as:

\[ \rho = |\psi\rangle \langle \psi| \]

where
\(\rho_{ij} = \langle \phi_i | \psi \rangle \langle \psi | \phi_j \rangle\)

The expectation value of an operator \(\hat{A}\) is given by:

\(\langle \hat{A} \rangle = Tr(\rho \hat{A})\\Proof:\\\)
\(\langle \hat{A} \rangle = \langle \psi | \hat{A} | \psi \rangle = \sum_{i,j} \rho_{ij} \langle \phi_i | \hat{A} | \phi_j \rangle \\
= \sum_{i} \rho_{i} \langle \phi_i | \hat{A} \sum_{j} |\phi_j\rangle \langle \phi_j| \phi_j \rangle = \sum_{i} \rho_{i} \langle \phi_i | \hat{A} | \phi_i \rangle \\
= Tr(\rho \hat{A})\)

The density operator can be written as:

\[\rho = \frac{1}{2} (I + \vec{r} \cdot \vec{\sigma})\]

where \(\vec{r} = (r, \theta, \phi)\)

\[Tr(\rho) = 1\]

Purity of a state is given by:
\[Tr(\rho^2) = \sum_{i} \rho_{ii}^2 \leq 1\] if \(Tr(\rho^2) = 1\) then
the state is pure else it is mixed. Purity = 1 - \(Tr(\rho^2)\)

Von Neumann entropy is given by: \[S(\rho) = -Tr(\rho \log_2 \rho)\]
\(S(\rho) = 0\) for pure states and \(S(\rho) > 0\) for mixed states.

\subsection*{Measurement}\label{measurement}
\addcontentsline{toc}{subsection}{Measurement}

Probability of measuring a state \(|\psi\rangle\) in the state
\(|\phi\rangle\) is given by:

\[P(\phi) = |\langle \phi | \psi \rangle|^2\]

It can be also given using the density operator as:

\[P(\phi) = \langle \phi | \rho | \phi \rangle = Tr(\rho |\phi\rangle \langle \phi|)\]

\[ P(\phi) = Tr(\rho \Pi_\phi) \]

where \(\Pi_\phi = |\phi\rangle \langle \phi|\) is the projection
operator.

\[\sum_{\phi} \Pi_\phi = I\]

\[\Pi_\phi \Pi_\psi = \delta_{\phi\psi} \Pi_\phi\]

The expectation value of an operator \(\hat{A}\) is given by:

\[\langle \hat{A} \rangle = \sum_{\phi} P(\phi) \langle \phi | \hat{A} | \phi \rangle = Tr(\rho \hat{A})\]

The density operator after measurement is given by:

\[ \rho' = \frac{\Pi_\phi \rho \Pi_\phi}{P(\phi)} = \frac{\Pi_\phi \rho \Pi_\phi}{Tr(\rho \Pi_\phi)} \]

\subsection*{Positive Operator Valued Measure
(POVM)}\label{positive-operator-valued-measure-povm}
\addcontentsline{toc}{subsection}{Positive Operator Valued Measure
(POVM)}

A POVM is a set of operators \(\{E_i\}\) such that \(E_i \geq 0\) and
\(\sum_i E_i = I\) where \(I\) is the identity operator.

The probability of measuring a state \(|\psi\rangle\) in the state
\(|\phi\rangle\) is given by:

\[P(\phi) = \sum_i \langle \phi | E_i | \phi \rangle = Tr(\rho E_i)\]

\subsection*{Multiple Quantum Systems}\label{multiple-quantum-systems}
\addcontentsline{toc}{subsection}{Multiple Quantum Systems}

2 level system =\textgreater{} qubit

d level system =\textgreater{} qudit

The Hilbert space of a composite system is the tensor product of the
Hilbert spaces of the individual systems.

\[\mathcal{H} = \mathcal{H}_1 \otimes \mathcal{H}_2\]

The basis vectors of the composite system are given by:

\[|\phi_i\rangle \otimes |\psi_j\rangle = |\phi_i\psi_j\rangle\]

The basis vectors of the composite system are orthonormal if the basis
vectors of the individual systems are orthonormal.

\[\langle \phi_i | \phi_j \rangle = \delta_{ij}\ ;\ \langle \psi_i | \psi_j \rangle = \delta_{ij}\]

The basis vectors of the composite system are complete if the basis
vectors of the individual systems are complete.

\[\sum_{i,j} |\phi_i\rangle \langle \phi_i | \otimes |\psi_j\rangle \langle \psi_j | = I\]

The density operator of the composite system is given by:

\[\rho = \rho_1 \otimes \rho_2\]

The expectation value of an operator \(\hat{A}\) is given by:

\[\langle \hat{A} \rangle = Tr(\rho \hat{A}) = Tr(\rho_1 \otimes \rho_2 \hat{A}_1 \otimes \hat{A}_2) = Tr(\rho_1 \hat{A}_1) Tr(\rho_2 \hat{A}_2)\]

Generic States:

\[|\psi\rangle_a = \sum_{i} c_{i} |\phi_i\rangle\ ;\ |\psi\rangle_b = \sum_{j} d_{j} |\psi_j\rangle\
where\ c_{ij} \in \mathbb{C}\]

\[ |\psi\rangle_{ab} = \sum_{i,j} c_{ij} |\phi_i\rangle \otimes |\psi_j\rangle \]
e.g.~\(|\psi\rangle_{ab} = \frac{1}{2} (|00\rangle + |01\rangle + |10\rangle + |11\rangle)\\
= \frac{1}{\sqrt{2}} (|0\rangle_a + |1\rangle_a) \otimes \frac{1}{\sqrt{2}} (|0\rangle_b + |1\rangle_b)\\
= |\psi\rangle_a \otimes |\psi\rangle_b\)

Non-generic States (non-separable):
\[|\psi\rangle_{ab} \neq |\psi\rangle_a \otimes |\psi\rangle_b\]

e.g.~\(|\psi\rangle_{ab} = \alpha|00\rangle + \beta|11\rangle\)

They are called entangled states.

\subsection*{Bell States}\label{bell-states}
\addcontentsline{toc}{subsection}{Bell States}

The Bell states are given by:

\[ |\Phi^+\rangle = \frac{1}{\sqrt{2}} (|00\rangle + |11\rangle)\]
\[ |\Phi^-\rangle = \frac{1}{\sqrt{2}} (|00\rangle - |11\rangle)\]
\[ |\Psi^+\rangle = \frac{1}{\sqrt{2}} (|01\rangle + |10\rangle)\]
\[ |\Psi^-\rangle = \frac{1}{\sqrt{2}} (|01\rangle - |10\rangle)\]

\(|\Phi^+\rangle\) and \(|\Phi^-\rangle\) are symmetric under exchange
of qubits.

\(|\Psi^+\rangle\) and \(|\Psi^-\rangle\) are anti-symmetric under
exchange of qubits.

Probability of measuring a Bell state \(|\Phi^+\rangle\) in the state
\(|00\rangle\) is given by: \textgreater{}
\(P(00) = |\langle 00 | \Phi^+ \rangle|^2 = \frac{1}{2}\)

This randomness of Bell states is used in Quantum Technologies.

Bell states are eigenstates of the Pauli operators of the composite
systems with eigenvalues = +-1

Entanglement persistes in different basis

\subsection*{Entanglement
Quantification}\label{entanglement-quantification}
\addcontentsline{toc}{subsection}{Entanglement Quantification}

\subsubsection*{Partial Projection}\label{partial-projection}
\addcontentsline{toc}{subsubsection}{Partial Projection}

Partial projection of a state \(|\psi\rangle\) is given by:

\[ \Pi_a = \sum_{i} |\phi_i\rangle \langle \phi_i | \otimes I_b \]

\subsubsection*{Partial Trace}\label{partial-trace}
\addcontentsline{toc}{subsubsection}{Partial Trace}

Partial trace of a state \(|\psi\rangle\) is given by:

\[ \rho_a = Tr_b(|\psi\rangle \langle \psi|) = \sum_{i} \langle \phi_i | \psi \rangle \langle \psi | \phi_i \rangle \]
\[ \rho_b = Tr_a(|\psi\rangle \langle \psi|) = \sum_{j} \langle \psi_j | \psi \rangle \langle \psi | \psi_j \rangle \]

\subsection*{3 Qubit System}\label{qubit-system}
\addcontentsline{toc}{subsection}{3 Qubit System}

The basis vectors of the composite system are given by:

\[|\phi_i\rangle \otimes |\psi_j\rangle \otimes |\chi_k\rangle = |\phi_i\psi_j\chi_k\rangle\]

\subsubsection*{GHZ State}\label{ghz-state}
\addcontentsline{toc}{subsubsection}{GHZ State}

The GHZ state is given by:

\[ |\psi\rangle_{GHZ} = \frac{1}{\sqrt{2}} (|000\rangle + |111\rangle) \]

The state is maximally entangled.

\subsubsection*{W State}\label{w-state}
\addcontentsline{toc}{subsubsection}{W State}

The W state is given by:

\[ |\psi\rangle_{W} = \alpha |001\rangle + \beta |010\rangle + \gamma |100\rangle \]

\subsection*{Entanglement Entropy / Von Neumann
Entropy}\label{entanglement-entropy-von-neumann-entropy}
\addcontentsline{toc}{subsection}{Entanglement Entropy / Von Neumann
Entropy}

The entanglement entropy of a state \(|\psi\rangle\) is given by:

\[ S(\rho_a) = -Tr(\rho_a \log_2 \rho_a) \]

where \(\rho_a = Tr_b(|\psi\rangle \langle \psi|)\)

\subsection*{Schmidt Decomposition}\label{schmidt-decomposition}
\addcontentsline{toc}{subsection}{Schmidt Decomposition}

The Schmidt decomposition of a state \(|\psi\rangle\) is given by:

\[ |\psi\rangle = \sum_{i} \sqrt{\lambda_i} |\phi_i\rangle \otimes |\psi_i\rangle \]

where \(\lambda_i\) are the eigenvalues of \(\rho_a\) and
\(|\phi_i\rangle\) and \(|\psi_i\rangle\) are the eigenvectors of
\(\rho_a\) and \(\rho_b\) respectively.

\[\rho_a = \sum_{i} \lambda_i |\phi_i\rangle \langle \phi_i |\\
\rho_b = \sum_{i} \lambda_i |\psi_i\rangle \langle \psi_i |\]

\[ S(\rho_a) = -\sum_{i} \lambda_i \log_2 \lambda_i = S(\rho_b) \]

\bookmarksetup{startatroot}

\chapter*{Quantum Optics}\label{quantum-optics}
\addcontentsline{toc}{chapter}{Quantum Optics}

\markboth{Quantum Optics}{Quantum Optics}

\subsection*{Free Field Quantization}\label{free-field-quantization}
\addcontentsline{toc}{subsection}{Free Field Quantization}

\subsubsection*{Quantization of single mode electromagnetic
field}\label{quantization-of-single-mode-electromagnetic-field}
\addcontentsline{toc}{subsubsection}{Quantization of single mode
electromagnetic field}

Maxwell's equations: \[\nabla \cdot \vec{E} = 0\]
\[\nabla \cdot \vec{B} = 0\]
\[\nabla \times \vec{E} = -\frac{\partial \vec{B}}{\partial t}\]
\[\nabla \times \vec{B} = \mu_0 \epsilon_0 \frac{\partial \vec{E}}{\partial t}\]
where
\(\vec{E} = \vec{E}(\vec{r}, t)\ ;\ \vec{B} = \vec{B}(\vec{r}, t)\)

Field polarization along x-axis: \[\vec{E} = E_x(z, t) \hat{x}\ \]

Single mode field:
\[ \vec{E} = \vec{E}(z, t) = \sqrt{\frac{\hbar \omega}{2 \epsilon_0 V}} q(t)\sin(kz) \hat{x}\ \]

where \(q(t)\) has dimensions of length, \(V\) is the volume of the
cavity and \(\omega_m = cm\pi/L\) where \(c\) is the speed of light.
Also

\[ \vec{B} = \vec{B}(z, t) = \frac{\mu_0\epsilon_0}{k}\sqrt{\frac{\hbar \omega}{2 \epsilon_0 V}}\ p(t)\cos(kz) \hat{y}\ \]

where \(p(t) = \dot{q}(t)\) has dimensions of momentum.

We can write the Hamiltonian of the field as:

\[ H = \frac{1}{2} \int_V dV (\epsilon_0 E^2 + \frac{1}{\mu_0} B^2) = \frac{1}{2}(p^2 + \omega^2 q^2) \]

\subsubsection*{1D Harmonic Oscillator}\label{d-harmonic-oscillator}
\addcontentsline{toc}{subsubsection}{1D Harmonic Oscillator}

The Hamiltonian of a 1D harmonic oscillator is given by:

\[ H = \frac{1}{2}(p^2 + \omega^2 q^2) \]

where \(p\) and \(q\) are the momentum and position of the classical
oscillator.

\subsubsection*{Quantization of 1D Harmonic
Oscillator}\label{quantization-of-1d-harmonic-oscillator}
\addcontentsline{toc}{subsubsection}{Quantization of 1D Harmonic
Oscillator}

\[q\rightarrow \hat{q}\ ;\ p\rightarrow \hat{p}\]

we have the commutation relation: \[ [\hat{q}, \hat{p}] = i\hbar \]

The Hamiltonian of the 1D quantum harmonic oscillator is given by:

\[ \hat{H} = \frac{1}{2}(\hat{p}^2 + \omega^2 \hat{q}^2) \] where
\(\hat{p}, \hat{q}\) are Hermitian operators.

\subsection*{Creation and Annihilation
Operators}\label{creation-and-annihilation-operators}
\addcontentsline{toc}{subsection}{Creation and Annihilation Operators}

The creation and annihilation operators are defined as:

\[ \hat{a} = \frac{1}{\sqrt{2\hbar\omega}} (\omega \hat{q} + i\hat{p}) \]
\[ \hat{a}^\dagger = \frac{1}{\sqrt{2\hbar\omega}} (\omega \hat{q} - i\hat{p}) \]

The operators are non-Hermitian.

also
\[ \hat{q} = \sqrt{\frac{\hbar}{2\omega}} (\hat{a} + \hat{a}^\dagger) \]
\[ \hat{p} = i\sqrt{\frac{\hbar\omega}{2}} (\hat{a}^\dagger - \hat{a}) \]

The commutation relation is given by:

\[ [\hat{a}, \hat{a}^\dagger] = 1 \]

Putting the above relations in previous equation we get:

\[ \hat{E_x} = \sqrt{\frac{\hbar \omega}{\epsilon_0 V}} (\hat{a} + \hat{a}^\dagger) \sin(kz) \]
\[ \hat{B_y} = \sqrt{\frac{\hbar \omega}{\epsilon_0 V}} (\hat{a}^\dagger - \hat{a}) \cos(kz) \]

where
\(\mathcal{E} = \sqrt{\frac{\hbar \omega}{\epsilon_0 V}}, \mathcal{B} = \sqrt{\frac{\hbar \omega}{\epsilon_0 V}}\)
are the field per photon.

The Hamiltonian of the field is given by:

\[ \hat{H} = \frac{1}{2}(\hat{p}^2 + \omega^2 \hat{q}^2) = \hbar \omega (\hat{a}^\dagger \hat{a} + \frac{1}{2}) \]

\subsection*{Evolution of the Field}\label{evolution-of-the-field}
\addcontentsline{toc}{subsection}{Evolution of the Field}

The time evolution of the field is given by:

\[ i\hbar \frac{\partial}{\partial t}\hat{a} = [\hat{a}, \hat{H}] \]
\[ \implies \frac{\partial}{\partial t}\hat{a} = -i\omega \hat{a} \]

The solution of the above equation is given by:

\[ \hat{a}(t) = \hat{a}(0) e^{-i\omega t} \]
\[ \hat{a}^\dagger(t) = \hat{a}^\dagger(0) e^{i\omega t} \]

\[\hat{E_x} = \mathcal{E}(\hat{a}(0)e^{-i\omega t} + \hat{a}^\dagger(0)e^{i\omega t}) \sin(kz)\]

\subsection*{Fock States}\label{fock-states}
\addcontentsline{toc}{subsection}{Fock States}

The Fock states are the eigenstates of the number operator
\(\hat{a}^\dagger \hat{a}\).

\[ \hat{a}^\dagger \hat{a} |n\rangle = n |n\rangle \] where \(n\) is the
number of photons in the state \(|n\rangle\) and
\(\hat{a}^\dagger \hat{a} = n\) is the photon number operator in the
mode.

\bookmarksetup{startatroot}

\chapter*{pQuantum States}\label{pquantum-states}
\addcontentsline{toc}{chapter}{pQuantum States}

\markboth{pQuantum States}{pQuantum States}

\bookmarksetup{startatroot}

\chapter*{Sources of pQuantum States}\label{sources-of-pquantum-states}
\addcontentsline{toc}{chapter}{Sources of pQuantum States}

\markboth{Sources of pQuantum States}{Sources of pQuantum States}

\bookmarksetup{startatroot}

\chapter*{Detection of Quantum Light}\label{detection-of-quantum-light}
\addcontentsline{toc}{chapter}{Detection of Quantum Light}

\markboth{Detection of Quantum Light}{Detection of Quantum Light}

\bookmarksetup{startatroot}

\chapter*{pQuantum Communication}\label{pquantum-communication}
\addcontentsline{toc}{chapter}{pQuantum Communication}

\markboth{pQuantum Communication}{pQuantum Communication}

\bookmarksetup{startatroot}

\chapter*{pQuantum Computation}\label{pquantum-computation}
\addcontentsline{toc}{chapter}{pQuantum Computation}

\markboth{pQuantum Computation}{pQuantum Computation}

\bookmarksetup{startatroot}

\chapter*{pQuantum Metrology}\label{pquantum-metrology}
\addcontentsline{toc}{chapter}{pQuantum Metrology}

\markboth{pQuantum Metrology}{pQuantum Metrology}

\bookmarksetup{startatroot}

\chapter*{References}\label{references}
\addcontentsline{toc}{chapter}{References}

\markboth{References}{References}

\phantomsection\label{refs}
\begin{CSLReferences}{1}{0}
\bibitem[\citeproctext]{ref-gerry05}
Gerry, Christopher C., and Peter L. Knight. 2005. \emph{Introductory
Quantum Optics}. Cambridge University Press.
\url{https://www.cambridge.org/highereducation/books/introductory-quantum-optics/FE693BC6459D0E6B48EFC}.

\end{CSLReferences}



\end{document}
